
\section{Project Implementation}

\begin{comment}
\begin{itemize}
    \item System Architecture: For technical projects, outline the architecture or structure of your system.
    \item Development Process: Discuss the steps you took to build, test, and deploy your solution.
    \item Technologies Used: Mention programming languages, frameworks, or tools employed.
\end{itemize}
\end{comment}





%#################################################################


\subsection{Computational Workflow Overview}

All experiments were conducted in an offline, notebook-based environment, with structured preprocessing, feature engineering, model training, and evaluation steps implemented as modular components. %Although no deployed system was constructed, the codebase was designed for reproducibility and extensibility, allowing future integration into production or real-time pipelines if desired.

\subsection{LSTM Model Architecture}

The Long Short-Term Memory (LSTM) models employed in this study are configured with $1...n$ layers, where the number of layers and their respective hidden units are determined by the model name. For example, \texttt{LSTM\_4} denotes a single-layer LSTM with 4 units, while \texttt{LSTM\_4\_2} denotes a two-layer LSTM with 4 units in the first layer and 2 units in the second layer.

Each model terminates with a dense output layer using a sigmoid activation function to map the final hidden state to a binary classification. Dropout regularization with a rate of 0.3 is applied after each LSTM layer to mitigate overfitting. Training is performed using the Adam optimizer and a binary focal loss function (equation (~\ref{eq:binary_focal_loss_function})), parameterized by $\gamma = 2.0$,  $\alpha$(equation (~\ref{eq:alpha})) and class weight calculations (equation (~\ref{eq:class_weight}))  to address class imbalance. 
Model performance is evaluated using the Area Under the Receiver Operating Characteristic Curve (AUC-ROC)


%both the Area Under the Receiver Operating Characteristic Curve (AUC-ROC) and the Area Under the Precision-Recall Curve (AUC-PR).

The following LSTM variants were implemented:

\begin{itemize}
\item \texttt{LSTM\_4}, \texttt{LSTM\_8}: Single-layer LSTM models with 4 or 8 hidden units.
\item \texttt{LSTM\_4\_4}, \texttt{LSTM\_8\_4}, \texttt{LSTM\_8\_8}: Two-layer LSTM models with corresponding hidden unit configurations.
\end{itemize}



%the value of alpha is computed from the class distribution in the training data as follows:

%$\alpha = \frac{\omega_1}{\omega_0 + \omega_1}$

\vspace{-24pt}

\begin{equation}
\alpha = \frac{\omega_1}{\omega_0 + \omega_1} %\tag{1}
\label{eq:alpha}
\end{equation}

where $\omega_0$ and $\omega_1$ are the class weights for the majority (no recession) and minority (recession) classes, respectively.

\vspace{-24pt}

\begin{equation}
\omega_c = \frac{n_{samples}}{n_{classes} \cdot n_c}
\label{eq:class_weight}
\end{equation}

\noindent
where:
\vspace{-9pt}
\begin{itemize} [itemsep = -6pt]
    \item $c = $ class 0 or class 1
    \item $\omega_c = $ weight for class $c$
    \item $n_{samples} = $ total number of training samples
    \item $n_{classes} = $ number of distinct classes
    \item $n_c = $ number of samples in class $c$
\end{itemize}

\vspace{-24pt}

\begin{equation}
L_{\text{focal}}(p_t) = -\alpha_t (1 - p_t)^\gamma \log(p_t)
\label{eq:binary_focal_loss_function}
\end{equation}

where $p_t$ is the predicted probability of the true class.


%and their respective hidden units are based on the model name (i.e LSTM\_4 means a LSTM with one layer that has 4 units and LSTM\_4\_2 mwans a LSTM with two layers that has 4 units in the first layer and 2 units in the second layer)

%Each model ends with a dense output layer that maps the final hidden state to a binary classification (sigmoid function). Dropout regularization with a rate of 0.3 is applied after each LSTM layer to mitigate overfitting. Training is performed using the Adam optimizer and a binary focal loss function parameterized by $\gamma = 2.0$ and $\alpha$ to address class imbalance


%LSTM\_models.extend(['LSTM\_4', 'LSTM\_8', 'LSTM\_4\_4', 'LSTM\_8\_4', 'LSTM\_8\_8'])

\subsection{Technologies Used}

\hl{List if relevant or maybe place math formulas here}

\begin{comment}
Key libraries used include:
\begin{itemize}
    \item \textbf{pandas} and \textbf{NumPy} for data manipulation and numerical computations;
    \item \textbf{scikit-learn} for classical machine learning models (e.g., Logistic Regression, Random Forest) and preprocessing utilities (e.g., SMOTE, train-test split);
    \item \textbf{xgboost} for gradient boosting classifiers;
    \item \textbf{imbalanced-learn} for handling class imbalance using ensemble methods and synthetic oversampling;
    \item \textbf{TensorFlow} and \textbf{Keras} for building, training, and tuning Long Short-Term Memory (LSTM) networks.
\end{itemize}
\end{comment}


diagram of LSTM and dive in on the explanation 