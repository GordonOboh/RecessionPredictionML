
\section{Project Implementation}

\begin{comment}
\begin{itemize}
    \item System Architecture: For technical projects, outline the architecture or structure of your system.
    \item Development Process: Discuss the steps you took to build, test, and deploy your solution.
    \item Technologies Used: Mention programming languages, frameworks, or tools employed.
\end{itemize}
\end{comment}





%#################################################################


\subsection{Computational Workflow Overview}

All experiments were conducted in an offline, notebook-based environment, with structured preprocessing, feature engineering, model training, and evaluation steps implemented as modular components. %Although no deployed system was constructed, the codebase was designed for reproducibility and extensibility, allowing future integration into production or real-time pipelines if desired.

\subsection{Technologies Used}

\hl{List if relevant or maybe place math formulas here}

\begin{comment}
Key libraries used include:
\begin{itemize}
    \item \textbf{pandas} and \textbf{NumPy} for data manipulation and numerical computations;
    \item \textbf{scikit-learn} for classical machine learning models (e.g., Logistic Regression, Random Forest) and preprocessing utilities (e.g., SMOTE, train-test split);
    \item \textbf{xgboost} for gradient boosting classifiers;
    \item \textbf{imbalanced-learn} for handling class imbalance using ensemble methods and synthetic oversampling;
    \item \textbf{TensorFlow} and \textbf{Keras} for building, training, and tuning Long Short-Term Memory (LSTM) networks.
\end{itemize}
\end{comment}