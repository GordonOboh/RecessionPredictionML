
\section{Methodology}
\begin{itemize}
    \item Approach/Method Used: Describe the methodology or approach you used for your project (e.g., qualitative, quantitative, experimental, case study).
    \item Tools and Techniques: Detail any tools, software, or technologies you used.
    \item Data Collection: Explain how you gathered data (e.g., surveys, experiments, secondary data).
    \item Data Analysis: Describe the analytical techniques you applied (e.g., statistical analysis, machine learning, content analysis).
\end{itemize}




This study extends prior work such as \textcite{joshi2020forecasting}, who employed a single LSTM architecture to forecast the 10-year minus 3-month Treasury yield spread. While Joshi’s application of LSTM demonstrates the potential of deep learning models in economic time series forecasting, it is limited in architectural variation and model tuning. In contrast, this study explores \textbf{multiple LSTM configurations}, including both shallow and stacked architectures, to assess model sensitivity and improve generalization. Additionally, whereas Joshi’s approach focused primarily on the yield spread as a univariate forecasting target, this study integrates the LSTM models into a broader classification framework aimed at directly predicting recession probabilities. This allows for a more practical and policy-relevant application of deep learning in macro-financial forecasting.