

\section{Literature Review}

\begin{comment}
    

\begin{itemize}
    \item Relevant Studies: Summarize the key literature in your field that informs your project.
    \item Gaps in Research: Identify where there is a lack of information or unaddressed problems in the current literature.
    \item Theoretical Framework: Discuss the theories or models you used to guide your project (if applicable).
\end{itemize}
\end{comment}

\subsection{Relevant Studies}

According to \textcite{NBERRecession}
A recession is generally defined as a significant decline in economic activity that lasts for an few months. This decline is typically visible in real GDP, real income, employment, industrial production, and wholesale-retail sales.

Translating this qualitative definition into a quantitative forecasting problem has led researchers to examine the predictive content of financial indicators. %But how do we express this mathematically?
The study by \textcite{bauer2018economic} highlights the predictive power of the term spread (10-year minus 1-year Treasury rates), they assert that inverted yield curves have preceded all U.S. recessions over the past 60 years, with only one false positive. Their empirical analysis confirms that a negative spread reliably signals elevated recession probability, even after accounting for potential structural changes in the interest rate environment.
Similarly, \textcite{aramonte2019yield} reinforce the historical reliability of the 10-year minus 3-month Treasury yield spread, noting that every U.S. recession since 1973 was preceded by an inversion of this spread. %However, they caution that recent distortions—such as exceptionally low term premia caused by central bank asset purchases—may complicate the interpretation of inverted yield curves.
Nonetheless, both studies converge on a key point: despite evolving macro-financial conditions, the inversion of the yield curve remains one of the most consistent and early indicators of impending recessions \parencite{bauer2018economic, aramonte2019yield}.


%we will draw upon the knowledge of \textcite{aramonte2019yield} and  \textcite{bauer2018economic}. The Study from \textcite{bauer2018economic} highlights the predictive power of the term spread (10-year minus 1-year Treasury rates), asserting that inverted yield curves have preceded all U.S. recessions over the past 60 years, with only one false positive. While the study from \textcite{aramonte2019yield} reinforces the historical reliability of the 10-year minus 3-month Treasury yield spread as a leading recession indicator in the U.S., noting that an inversion has preceded every recession since 1973. Despite the current low interest rate environment, the term spread’s signal remains undiminished, and a negative spread should still be viewed as a serious warning of an impending downturn \parencite{bauer2018economic}.




\begin{comment}
    

\subsection{Relevant Studies}

%According to \textcite{NBERRecession}, a recession is generally defined as a significant decline in economic activity spread across the economy, lasting more than a few months. This decline is typically visible in real GDP, real income, employment, industrial production, and wholesale-retail sales.

Translating this qualitative definition into a quantitative forecasting problem has led researchers to examine the predictive content of financial indicators, particularly the yield curve. Among the most studied measures is the \textbf{term spread}, which captures the difference between long-term and short-term interest rates.

%The study by \textcite{bauer2018economic} highlights the predictive power of the term spread—specifically, the difference between the 10-year and 1-year Treasury yields—asserting that inverted yield curves have preceded all U.S. recessions over the past 60 years, with only one false positive. Their empirical analysis confirms that a negative spread reliably signals elevated recession probability, even after accounting for potential structural changes in the interest rate environment.

%Similarly, \textcite{aramonte2019yield} reinforce the historical reliability of the 10-year minus 3-month Treasury yield spread, noting that every U.S. recession since 1973 was preceded by an inversion of this spread. However, they caution that recent distortions—such as exceptionally low term premia caused by central bank asset purchases—may complicate the interpretation of inverted yield curves.

%Nonetheless, both studies converge on a key point: despite evolving macro-financial conditions, the inversion of the yield curve remains one of the most consistent and early indicators of impending recessions \parencite{bauer2018economic, aramonte2019yield}.
\end{comment}




%\newpage

\subsection{Gaps in Research}

\begin{comment}
\begin{itemize}
    \item Limited Focus on Alternative Yield Spreads
    \item Underexplored Use of Daily Frequency Data
    \item Insufficient Integration of Machine Learning with Macroeconomic Interpretation
    \item Neglect of Class Imbalance in Recession Prediction
    \item Lack of Comparative Evaluation Across Modeling Techniques
\end{itemize}
\end{comment}

%While the 10y-3m spread is a well established recession predictor and used by \textcite{NYFedYieldCurve} %\parencite{NYFedYieldCurve}
%, alternative spreads like 10y-2y remain under researched.

%\subsubsection{Limited Focus on Alternative Yield Spreads}

Despite the extensive body of research linking the yield curve to recession forecasting, several critical gaps remain. First, there is a limited focus on alternative yield spreads; the majority of studies concentrate on the 10-year minus 3-month or 1-year Treasury spread,
overlooking other combinations such as GS10–DGS2 or GS10–DGS3MO that may offer %earlier or 
more stable predictive signals.
Second, the use of high-frequency data such as daily or weekly averages is relatively
rare, as most prior work relies on monthly or quarterly averages, potentially missing high-frequency dynamics crucial for near real-time forecasting.
Third, there is an insufficient integration of machine learning with macroeconomic datasets. % interpretation.
Fourth, many studies neglect the issue of class imbalance inherent in recession data. Recession periods are rare relative to expansions, yet performance metrics are often reported without adjustment for this imbalance, potentially leading to misleading conclusions.
Finally, there is a lack of comparative evaluation across modeling techniques; few studies evaluate both classical econometric models and modern machine learning approaches within the same empirical framework. This leaves open the question of which approaches are most robust across different data conditions and policy regimes.

\begin{comment}
    

\subsection{Gaps in Research}

%Despite the extensive body of literature linking the yield curve to recession forecasting, several critical research gaps remain. 

%First, there is a \textbf{limited focus on alternative yield spreads}. Most studies emphasize the 10-year minus 3-month or 10-year minus 1-year Treasury spreads, while neglecting variants such as GS10--DGS2 or GS10--DGS3MO. These alternatives may yield more stable or timely signals but remain underexplored in empirical work.

%Second, the \textbf{use of high-frequency data} such as daily or weekly averages is relatively rare. Existing studies tend to rely on monthly or quarterly observations, which may obscure important short-term dynamics necessary for real-time policy or investment decisions.

%Third, there is an \textbf{insufficient integration of machine learning with macroeconomic interpretation}. While machine learning models such as LSTM and XGBoost are gaining popularity, they are often used as black boxes, with limited effort to connect their predictions to underlying economic theory or structural dynamics.

%Fourth, many studies \textbf{neglect the issue of class imbalance} inherent in recession data. Recession periods are rare relative to expansions, yet performance metrics such as accuracy are often reported without adjustment for this imbalance, potentially leading to misleading conclusions.

Finally, there is a \textbf{lack of systematic comparison across modeling techniques}. Few studies evaluate both classical econometric models (e.g., ARIMA, probit) and modern machine learning approaches (e.g., ensemble methods, gradient boosting) within the same empirical framework. This leaves open the question of which approaches are most robust across different data conditions and policy regimes.
\end{comment}


%########

%Despite the extensive body of research linking the yield curve to recession forecasting, several critical gaps remain. First, there is a \textbf{limited focus on alternative yield spreads}; the majority of studies concentrate on the 10-year minus 3-month or 1-year Treasury spread, overlooking other combinations such as GS10--DGS2 or GS10--DGS3MO that may offer earlier or more stable predictive signals. Second, the \textbf{use of daily frequency data remains underexplored}, as most prior work relies on monthly or quarterly averages, potentially missing high-frequency dynamics crucial for real-time forecasting. Third, there is an \textbf{insufficient integration of machine learning with macroeconomic interpretation}; while models like LSTM and random forests are increasingly used, they are often treated as black boxes with little effort to reconcile their predictions with economic theory. Fourth, many studies \textbf{neglect the problem of class imbalance}, assuming equal distribution between recession and non-recession periods—this undermines the validity of common performance metrics like accuracy. Lastly, there is a \textbf{lack of comparative evaluation across modeling techniques}; few studies systematically compare classical econometric methods (e.g., ARIMA, probit) with modern machine learning approaches (e.g., XGBoost, ensemble classifiers) to assess which are most robust in practical forecasting scenarios.

\newpage

\subsection{Theoretical Framework}
This study is grounded in the theoretical relationship between the yield curve and the business cycle. According to the \textit{Expectations Hypothesis} of the term structure of interest rates, long-term interest rates reflect the average of expected future short-term interest rates. When investors anticipate economic slowdowns, they expect future short-term rates to fall. This shift in expectations can cause long-term yields to fall below short-term rates, resulting in an inverted yield curve. \hl{Historically, such inversions have preceded U.S. recessions, making the yield curve a widely recognized leading indicator of economic downturns \parencite{estrella1998predicting}.}

This study aims to use several modeling approaches; Logistic Regression a traditional econometric model. In addition, tree-based ensemble models such as Balanced Random Forest and Easy Ensemble Classifier are employed to account for non-linearities and imbalanced class distributions inherent in recession data. Finally, the study incorporates various architectures of Long Short-Term Memory (LSTM) networks. 



\subsection{Theoretical Framework}

%This study is grounded in the theoretical relationship between the yield curve and the business cycle. According to the \textit{Expectations Hypothesis} of the term structure of interest rates, long-term interest rates represent the average of expected future short-term rates. When investors anticipate an economic slowdown, they expect future short-term interest rates to decline. This shift in expectations can cause long-term yields to fall below short-term rates, resulting in an inverted yield curve. Historically, such inversions have preceded U.S. recessions, making the yield curve a widely recognized leading indicator of economic downturns \parencite{estrella1998predicting}.

%To operationalize this theoretical insight, the study employs a range of modeling approaches. First, \textbf{Logistic Regression}, a traditional econometric model, is used to estimate the probability of a recession based on the slope of the yield curve. This aligns with classical macro-financial literature that models binary outcomes, such as recession occurrence, using probit or logit specifications.

%Second, the study incorporates \textbf{tree-based ensemble models}, including the \textit{Balanced Random Forest} and the \textit{Easy Ensemble Classifier}. These models are well-suited for capturing non-linear interactions among predictors and handling class imbalance—an inherent challenge in recession prediction, given the rarity of recessionary periods.

%Finally, the study integrates \textbf{Long Short-Term Memory (LSTM)} networks, a type of recurrent neural network (RNN) capable of learning long-range dependencies in sequential data. LSTMs are particularly useful in modeling time series such as yield spreads, as they can capture temporal patterns and structural shifts that are difficult to model using static or linear methods. 

The inclusion of LSTMs reflects a growing trend in macroeconomic forecasting to complement traditional statistical tools with machine learning architectures capable of processing complex, high-frequency data.



%, XGB Classifier, Balanced Random Forest Classifier, Easy Ensemble Classifier and LSTM

 
%Finally, the study incorporates various architectures of \textbf{Long Short-Term Memory (LSTM)} networks. LSTMs are a type of recurrent neural network capable of learning long-range dependencies in sequential data, making them particularly effective for modeling temporal patterns in financial time series. These models are designed to capture both short-term fluctuations and longer-term cyclical signals in the yield spread, offering a data-driven complement to more traditional econometric methods.

\begin{comment}
    
\subsection*{Theoretical Framework}

%This study is grounded in the theoretical relationship between the \textbf{yield curve} and the \textbf{business cycle}, particularly the use of the \textbf{term spread} as a leading economic indicator. According to the \textit{Expectations Hypothesis} of the term structure of interest rates, long-term interest rates reflect the average of expected future short-term rates. When investors anticipate economic slowdowns, they expect future short-term rates to fall, leading to a decline in long-term yields and potentially an \textit{inverted yield curve}. Such inversions—especially those involving the 10-year minus 3-month or 2-year Treasury rates—have historically preceded U.S. recessions.

%The project also draws from the \textbf{Liquidity Premium Theory}, which extends the Expectations Hypothesis by incorporating a risk premium for holding long-term bonds. This premium may vary with market conditions, introducing complications in interpreting yield curve signals, particularly during periods of unconventional monetary policy.

%To operationalize these concepts, the study uses several modeling approaches. From the econometric tradition, \textbf{logistic regression} is used to estimate the probability of a recession based on term spread inputs, aligning with previous work by Estrella and Mishkin (1998). In addition, \textbf{tree-based ensemble models} such as \textit{Balanced Random Forest} and \textit{Easy Ensemble Classifier} are employed to account for non-linearities and imbalanced class distributions inherent in recession data.

%Finally, the use of \textbf{SMOTE (Synthetic Minority Over-sampling Technique)} and \textbf{Random UnderSampling} within the modeling pipeline reflects an applied commitment to addressing real-world data challenges—particularly the rarity of recession events. These modeling choices are not only practical but consistent with recent advances in predictive macroeconomics that combine traditional theory with machine learning to enhance forecasting robustness.
\end{comment}


