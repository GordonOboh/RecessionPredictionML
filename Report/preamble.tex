\documentclass[stu,12pt,floatsintext,article]{apa7}

\hyphenpenalty=10000
\exhyphenpenalty=10000

\usepackage{comment}
\usepackage[american]{babel}
\usepackage{amsmath}
\usepackage{csquotes} 
\usepackage[style=apa,sortcites=true,sorting=nyt,backend=biber]{biblatex}
\DeclareLanguageMapping{american}{american-apa} % Gotta make sure we're patriotic up in here. Seriously, though, there can be local variants to how citations are handled, this sets it to the American idiosyncrasies 
\addbibresource{bibliography.bib} 
\usepackage[T1]{fontenc} 
%\usepackage{mathptmx} % This is the Times New Roman font.
\usepackage{newtxtext,newtxmath}


\usepackage{graphicx}
\graphicspath{ {./images/} }
\newcommand{\comma}{,}
\usepackage[final]{pdfpages}
\usepackage{enumitem}
\usepackage{pdflscape}
\usepackage{setspace}
%\usepackage[margin=0.1in]{geometry}
%\usepackage{luacolor} % Required to use the lua-ul \highLight command 
%\usepackage{lua-ul}
%\baselineskip=28pt
%between 27-29 is the closest to windows spacing
\usepackage{hyperref}
\usepackage{url}
%\usepackage{float}